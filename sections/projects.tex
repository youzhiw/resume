\documentclass[a4paper,12pt]{article}
\usepackage{TLCresume}
\begin{document}
\section{NLP project} 
\subsection{{Creating robust datasets for LLM evaluation \hfill\textnormal{(New York, NY)}  May 2024 -- August 2024}}
\begin{zitemize}
\item Coauthored a paper for EMNLP conference
\item Parameterized 1/3 of a math questions dataset (GSM8K) to introduce randomness to better evaluate LLM’s reasoning ability
    \item Used python code to generate 400 questions and answers for evaluating LLMs
\end{zitemize}

\section{Deep Learning projects}
\subsection{Schizophrenia classification	\hfill \textnormal{(New York, NY)} Jan 2024 -- May 2024}
\begin{zitemize}
\item Trained a VGG11 CNN model from scratch on 3D MRI brain images that achieved AROC of 0.87
\item Utilized GradCAM from VGG11 model to visualize regions of the brain that cause schizophrenia
% \item Fine tuned SWIN T transformer model to classify schizophrenia on 8000+ 2D image slices 
\end{zitemize}
\subsection{Cancer cell detection and classification \hfill	\textnormal{(New York, NY)} Jan 2024 -- May 2024}
\begin{zitemize}
% \item Advised by Prof. Corey Toler-Franklin in the GILM lab	
\item Fine tuned SWIN T transformer model to detect position of cancer cell in a image that achieves precision of 0.63
\item Trained Neural Network model using AWS nodes
\end{zitemize}

\subsection{ Multiple Sclerosis classification using Brain MRI	\hfill \textnormal{(New York, NY)} Sep 2023 -- Jan 2024}
\begin{zitemize}
% \item Advised by Dr. Korhan Buyuturkoglu and Albert Boulanger at CUIMC	
\item Wrote Python scripts using PyRadiomics to run radiomics analysis for 100 brain MRIs using clusters
\item Wrote Bash scripts that runs brain MRI segmentation using FreeSurfer
\item Wrote Python scripts using PyTorch and MONAI for classifying Multiple Sclerosis based on radiomics analysis that achieves 70\% accuracy
\end{zitemize}
\subsection{Earthquake data collection and classification \hfill\textnormal{(Santa Barbara, CA)}	Jul 2022 -- Jun 2023}
\begin{zitemize}
\item Received Applied math Major Distinction award from the University of California, Santa Barbara for my work
\item Collected and cleaned large collections of seismic signals (over 100,000 waveforms) using Python and ObsPy
\item Improved high-quality signal identification time by 5 times through restructuring the previous data processing pipeline in JupyterNotebooks
\item Labeled 200+ high-quality Earthquake signals that can be used to train ML algorithms 
\item Tested 5 different Machine Learning algorithms (KNN, LDA/QDA, SVMs and DCNN) for classifying high quality seismic wave signals
\end{zitemize}
\section{other coding projects}
\subsection{HTTP 1.0 Web server \hfill	Sep 2023 -- Dec 2023}
\begin{zitemize}
\item Wrote a HTTP 1.0 web server using C and socket APIs with image hosting and an interactive search module that uses a linked list to search through a given database.
\end{zitemize}
\end{document}
